\section{Introduction} \label{sec:intro}

\subsection{Physical system}

The physical system that I want to study is the altitude control of a
fixed wing aircraft.

\subsection{Motivation for study}

This is related to my research as I delve into various flight control
software. I think it will be good for me to have a deeper
understanding of how they work.

\subsection{Description of system states}

I am considering using the full 12 states of the aircraft which
include:

\begin{enumerate}
\item Positions in the inertia frame: $x$, $y$, $z$ 
\item Velocities in body frame: $v_{bx}$, $v_{by}$, $v_{bz}$
\item Roll, pitch, yaw rates: $p$, $q$, $r$
\item Euler angles: $\phi$, $\theta$, $\psi$
\end{enumerate}

\subsection{Description of sensor data used for estimation}

If we assume all states can be sensed directly, one potentially
interesting route would be to add instrumentation into ardupilot to
record the values of the state variables. Pick a representative
scenario and derive the data. In fact, in one of the bug injection
experiments I had, one of the bugs had a large pitch demand at one point
when the plane came in for a landing. Maybe in a later project when the states
are not all directly sensed I can bring in the sensors.

As for the sensors, I'm not sure about this yet, but I think the following would make
sense (basically the things available in Ardupilot)

\begin{enumerate}
\item IMU (accelerometer, gyros)
\item GPS
\item Barometer
\end{enumerate}

\subsection{Description of approach to develop a model}

I plan on using the following references to get the dynamics equations:

\begin{enumerate}
  \item Model-based fault diagnosis for aerospace systems: a survey
    \cite{marzat_model-based_2012}
  \item Aircraft Dynamics and Automatic Control \cite{mcruer2014aircraft}
\end{enumerate}
  
