\section{Analytical Methods} \label{sec:methods}

%% Description of the methods and key equations you use in your design
%% and performance assessment - including tools used to obtain a linear
%% model and tools used to design a controller. 

In order to place poles, I first looked at the step response of the
open-loop system, shown in Figure \ref{fig:openloop}.
\begin{figure}[h]
  \ig{scale=0.25}{747_openloopresponse.png}
  \caption{Open loop response of the Boeing 747 system to impulse.}
  \label{fig:openloop}
\end{figure}
The eigenvalues
of the Boeing 747 linearized model are: 0, $-0.0005 \pm 0.0675i$, $-0.375 \pm
0.8817i$. These eigenvalues mean that the system has a dominant pole
that does not decay since system stability is guaranteed only if
$Re(\lambda) > 0$. To ensure a 4 second rise time and a 2\% settling time of
12 seconds, I use the following equations:
\begin{align}
  t_r &\approx \frac{1.8}{\omega_n} \\
  t_s &= \frac{4.0}{\sigma} \\
  \omega_d &= \sqrt{\left( \omega_n^2 - \sigma^2 \right)} \\
  \lambda &= -\sigma \pm \omega_dj 
\end{align}
Plugging in $t_{r}$ and $t_{s}$ with 4 and 12, I solve for $\omega_n$
and $\sigma$, which is $0.45$ and $\frac{1}{3}$ respectively. Using
these two values I solve for $\omega_d \approx 0.0914$. From our
analytical solution, we see that the slow poles need to be placed at
$-0.3023 \pm 0.45i$. Since the fast poles need to be at least four
times faster, the other poles are located at: $-1.2092 \pm 0.0675i$
and $-1.2092 \pm 0.8817i$. Looking at the step response of the
closed-loop system, we see that the rise time and settling time of the
system meets our specifications. 
\begin{figure}[h]
  \ig{scale=0.25}{747_closedloopresponse.png}
  \caption{Closed loop response after pole placement of the Boeing 747 system to impulse.}
  \label{fig:closeloop}
\end{figure}
