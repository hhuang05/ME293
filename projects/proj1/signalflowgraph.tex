\section{Signal-flow Graph}

Signal-flow graphs were originally invented by Shannon and
later developed by Mason. The flow graph consist of nodes connected by
directed edges and is used to represent a set of linear
relationships. The origin of the term comes from the fact that this
type of analysis was being applied to electrical circuits. Before we
go further, it is useful to first define some terms.

\subsection{Definitions}

The basic element that makes up a signal-flow graph is a
\textit{branch}, which is a directed edge that connects between two
\textit{nodes}. Nodes represent important variables in the system. The
direction of the branch relates two nodes and creates a dependency
between them in terms of the signal flow. The relations between nodes
is written next to the branch and is either positive or negative to
indicate the type of feedback. A \textit{path} is made up of one or
more branches that allows a signal to flow from one node to
another. The following example taken from \cite{dorf_modern_2011}
illustrates these concepts.

\noindent Consider a system represented by two state equations:
\begin{align*}
  \dot x_1 &= -2x_1 + u \\
  \dot x_2 &= dx_1 - 3x_2 \\
\end{align*}
Given that $y = x_2$, we can rewrite these equations into the
state space form:

\begin{align*}
  \mathbf{\dot x} &=
  \begin{bmatrix}
    -2 & 0 \\
    d & -3 \\
  \end{bmatrix}
  \mathbf{x} +
  \begin{bmatrix}
    1 \\
    0 \\
  \end{bmatrix}
  u \\
  y &=
  \begin{bmatrix}
    0 & 1 \\
  \end{bmatrix}
  \mathbf{x} +
  [0] u
\end{align*}

This turns into the following signal-flow graph:

\begin{figure}[h]
  \ig{scale=0.4}{signalflowgraph1.eps}
  \caption{Signal-flow graph of the system of simultaneous equations.}
  \label{fig:signalflowgraph1}
\end{figure}

