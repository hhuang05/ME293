\section{Evaluation} \label{evaluation_1}

\par{~}

\hangindent = 0.35in\textbf{Sensing} - Poll and low-pass filter simulated noisy sensors

\hangindent = 0.35in\textbf{Communication} - Check for external broadcast, possibly a command from air traffic control

\hangindent = 0.35in\textbf{Control} - Update actuator commands based on proportional-derivative response to error between filtered sensor readings and reference trajectory

\hangindent = 0.35in\textbf{Alarm} - Verify that sensor readings are within the error tolerance

\par{~}

All tasks except for the Control task have a nominal period of 0.01
sec. The Control task executes with a nominal period of 1 second. The
Sensor task simulates two sensors readings, an altimeter and a
vertical GPS velocity. Both sensors are simulated subject to additive
Gaussian white noise. The measurement noise levels, as described by
their standard deviation, are set randomly at the beginning of each
simulation. The goal of varying sensor noise is to avoid over-tuning
algorithm parameters given model-to-model variations for real sensor
hardware. To this end, each simulation sets the altimeter-noise
standard deviation to a random value uniformly sampled on the range 5
- 15 m. Similarly, each simulation sets the velocity-noise standard
deviation to a random value uniformly sampled on the range 0.1 - 1.0 m/sec.

In all simulations, the aircraft maintains a constant horizontal
velocity of 120 m/sec. The aircraft vertical velocity is dependent on
the control-law response to randomized sensor noise and initial
conditions.  In particular, initial conditions for descent angle and
horizontal distance to touchdown point are selected randomly using
uniform sampling (on the range of 0$^\circ$-5$^\circ$ for descent
angle and on the range of 5 - 20 km for horizontal distance).

The custom autopilot simulator was instrumented to capture seven
signals for analysis by the proposed bug-detection algorithm.  These
signals included:  sensed altitude (m), sensed vertical velocity (m),
estimated vertical acceleration (m/sec$^2$), time interval between
iterations of the Sensor thread (msec), pitch angle ($^\circ$), total
time in descent (s), altitude for reference trajectory (m). These
instrumented $N=7$ variables were labeled $x_n$ in Section \ref{methods}.

\subsection{Continuous Descent and Missed Approach Scenarios}

\begin{figure}[h]
  \ig{scale=0.4}{landing_scenario.eps} 
  \caption{Landing scenario.}
  \label{fig:landing}
\end{figure}

The custom autopilot can be used to simulate two aircraft landing
scenarios, as depicted in Figure \ref{fig:landing}. One scenario
is a standard approach in which the aircraft follows a continuous
descent trajectory during the final phase of landing.  The other
scenario is a missed approach, an emergency situation that occurs, for
example in response to communication from air traffic controllers
about a runway incursion or in response to an alert from the
aircraft's Traffic Collision Avoidance System (TCAS). The missed
approach is simulated by abruptly changing the control reference from
a continuous descent trajectory to a continuous ascent trajectory. The
controller responds accordingly, triggering the aircraft to begin an
abrupt climb.

\subsection{Fault Injection Scenarios}
The proposed algorithm is aimed, in large part, at identifying
intermittent bugs, which appear under some circumstances but not
others.  For such bugs, normal processing occurs most of the time,
when the bug is not active; however, faulty processing occurs
sometimes, when the bug becomes active.  To exercise both these cases,
the source code is configured with a trigger that injects faults by
running buggy code.  Two software bugs are considered:  a unit
conversion bug and an integer overflow bug.  The first type of bug
famously caused the loss of the Mars Climate Orbiter \cite{wiki:mars}.
The second type of bug recently resulted in the grounding of the
Boeing 787 Dreamliner fleet  \cite{boeing_bug}.

In the case of the units conversion bug, a function fails to convert
the system clock time from units of milliseconds to seconds.  This bug occurs in
the Sensor task. The integer overflow bug occurs at a predicate where
the Alarm task should be executed. However, due to an integer
overflow, the Alarm task is prevented from executing, causing a silent
hazard if the bug suppresses the alarm under conditions when it
otherwise should trigger. The four quadrants in Figure \ref{fig:fourscenarios}
summarizes the four scenarios that result from the combination of
aircraft maneuvers and software fault cases.

\begin{figure}[h]
  \begin{subfigmatrix}{2}% number of columns
    \subfigure[Continous Descent - Normal]{\ig{}{normal_cont_desc.eps}}
    \subfigure[Continous Descent - Fault]{\ig{}{fault_cont_desc.eps}}
    \subfigure[Missed Approach - Normal]{\ig{}{normal_missed.eps}}
    \subfigure[Missed Approach - Fault]{\ig{}{fault_missed.eps}}
  \end{subfigmatrix}
  \caption{Two aircraft maneuvers and two software states (normal and
    faulty) results in a total of four scenarios used in training and
    evaluating our bug-detection algorithm. For each quadrant, the top
    graph plots the altitude and the bottom graph plots the vertical
    acceleration. The software fault pictured here is a units bug
    similar to the one that brought down the Mars Climbat Orbiter.}
  \label{fig:fourscenarios}
\end{figure}
