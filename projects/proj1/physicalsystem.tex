\section{Physical System} \label{sec:physys}

%% Description of the problem you are trying to solve and the states you
%% are trying to control; description of the performance goal(s) you seek
%% to achieve through feedback control.

The physical system is of a Boeing 747 aircraft and I want to
design a controller for the aircraft to maintain straight and level
flight. Due to the nonlinear coupling between airspeed and altitude,
slight changes in the pitch of the aircraft may lead to an eventual
large deviation from the cruising altitude. For this reason, this type
of controller is useful to control the aircraft to maintain cruising
altitude. I will focus on pole placement of the open loop system so as
to meet performance goals. In addition, I want to investigate how the
linear assumptions of the physical system responds to nonlinear
dynamics in simulation.

\subsection{System Description}

For aircraft, the kinematic equations of motion is written with
reference to two sets of axes, the inertial frame and the body
frame. The inertial frame lies outside the aircraft, usually it is a
reference fixed with respect to the earth. The body frame has its
origins at the center of mass of the aircraft, with the x-axis
pointing forward towards the nose, z-axis pointing down from the body
and the y-axis is perpenticular going out towards the right wing. An
example of these two sets of axes and the associated rotation is
pictured in figure ??.

\todo[inline]{NEED FIGURE HERE}

From these two sets of references, we define the kinematic equations
for an aircraft as follows:
\begin{align*}
  \begin{bmatrix}
    \dot x \\
    \dot y \\
    \dot z \\
  \end{bmatrix}
  &=
  T_{\psi}T_{\theta}T_{phi}
  \begin{bmatrix}
    u \\
    v \\
    w \\
  \end{bmatrix} \\
  \begin{bmatrix}
    \dot \phi \\
    \dot \theta \\
    \dot \psi \\
  \end{bmatrix}
  &=
  (\frac{1}{c \theta})
  \begin{bmatrix}
    c \theta & s\theta s \phi & s \theta c \phi \\
    0 & c \theta c \phi & -c \theta s \phi \\
    0 & s \phi & c \phi\\
  \end{bmatrix} 
  \begin{bmatrix}
    p \\
    q \\
    r \\
  \end{bmatrix} \\
\end{align*}
where
\begin{itemize}
\item $T_{\psi}, T_{\theta}, T_{\phi}$ are the appropriate rotation
matrices based on the three Euler angles: $\psi, \theta,
\phi$.
\item $x,y,z$ are the inertial frame coordinates.
\item $u,v,w$ are the body frame coordinates.
\item $p,q,r$ are the roll, pitch and yaw rates.
\end{itemize}

I made several assumptions to the aerodynamic forces to reduce the
order of the problem. The first assumption is that
the aerodynamic forces can be approximated by linear functions for
small perturbations. This means that longitudinal and lateral dynamics
can be de-coupled and treated independently. The second assumption is that this
system has thrust and the elevator as the only control inputs
into the system. With these two assumptions, the states in the system are: $[u, w, q, \theta, z,
  x]$, with control inputs $[\delta e$, $\delta t]$.

The third assumption is that the aircraft is in steady rectilinear
flight. This implies that $\dot u = \dot w = \dot q = \dot \theta =
0$. We can use the following linearized dynamic equations for $\dot x$
and $\dot z$:
\begin{align*}
  \dot x &= u \\
  \dot z &= -w + u_0 \theta \\
\end{align*}

Bryson \cite{bryson2015control} presents a linearized model of a
Boeing 747 at cruising altitude of 40,000 feet, with a velocity of
774 ft/sec. Adding the linear relationships of $\dot x$ and $\dot z$,
we have the following linearlized model in state space form:
\begin{align*}
  \begin{bmatrix}
    \dot u \\
    \dot w \\
    \dot q \\
    \dot \theta \\
    \dot z \\
    \dot x \\
  \end{bmatrix}
  &=
  \begin{bmatrix}
    -.003 & .039 & 0 & -.322 & 0 & 0 \\
    -.065 & -.319 & 7.74 & 0 & 0 & 0 \\
    .0201 & -.101 & -.429 & 0 & 0 & 0\\
    0 & 0 & 1 & 0 & 0 & 0 \\
    0 & -1 & 0 & 7.74 & 0 & 0 \\
    1 & 0 & 0 & 0 & 0 & 0 \\
  \end{bmatrix} 
  \begin{bmatrix}
    u \\
    w \\
    q \\
    \theta \\
    z \\
    x \\
  \end{bmatrix} \\
  &+  
  \begin{bmatrix}
    .01 & 1 \\
    -.18 & -.04 \\
    -1.16 & .598 \\
    0 & 0 \\
    0 & 0 \\
    0 & 0 \\
  \end{bmatrix}
  \begin{bmatrix}
    \delta_{e} \\
    \delta_{t} \\
  \end{bmatrix} \\
\end{align*}

\subsection{Performance Goals}

After pole placement, the system should have a rise time of 4 seconds and a 2\% settling time
of 12 seconds. Four seconds seems like a reasonable expected response
time to changing altitudes. The long settling time ensures that passengers
will not feel any jerking motion from controller compensation.
