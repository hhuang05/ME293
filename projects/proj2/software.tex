\section{Codebases} \label{autopilot}

We want to evaluate the efficacy of our bug detection system on
actual aviation software systems. However, commercial autopilots
from companies such Boeing or Airbus are proprietary and nearly
impossible to acquire. While NASA is quite open about \textit{some} of its
software, there has been a history of legal wrangling surrounding the
release of its flight software. Given all these environmental issues which
we can not control, it seemed most prudent for us to choose flight
software which was the best approximation of this domain. 

In the last five years, there has been an explosion of interest in the
use of Unmanned Aerial Vehicles (UAV), commonly called
\textit{drones}, for commercial purposes. UAV's were first used by
hobbiests, who mounted a camera on top of it to take aerial
photos. Not long after that, people changed the camera in favor of a
camcorder to record movies. Very soon, people had lots of uses for
UAV's, which ranged from land surveys to package delivery. We believe
that UAV autopilots serve as an excellent approximation for the domain
of avionics software. Even though UAV autopilots are written for soft
real-time applications, the open-source nature of the projects provide
a tremendous treasure trove of valuable defect data and associated fixes.

\subsection{UAV Autopilots} \label{uav_autopilots}
With the proliferation of UAV's, there are many open-source autopilot
projects. A list of them follows. They are all written in C and C++ and
all have well over hundred of thousands lines of code.

\begin{enumerate}
\item Ardupilot [C++] - SITL, automaked smoke tests
\item LibrePilot [C/C++] - Formerly Openpilot minial tests, SITL needs
  GUI
\item Cleanflight [C] - Comprehensive tests, no SITL.
  %% https://github.com/cleanflight/cleanflight
\item Paparazzi UAV [C] - Cross compile OCaml into C, SITL.
  %% https://wiki.paparazziuav.org/wiki/Main\_Page
\item Px4 [C/C++] - Comprehensive tests.
  %% https://github.com/PX4/Firmware
\item MatrixPilot [C] - minimal tests, SITL.
  %% https://code.google.com/p/gentlenav/
\item Minnesota Custom UAV [C] - Matlab simulations.
  %% http://www.uav.aem.umn.edu/wiki/Downloads
\end{enumerate}

\todo[inline]{Can we quantify how big is each code base}

The autopilot is reponsible for controlling the trajectory of the UAV
given a reference trajectory. Like many cyber-physical systems,
the autopilot has a series of inputs coming from sensors that monitor
its environment, some of these include: altitude, roll, pitch, yaw, position,
velocity. The output from the autopilot is an adjustment of the throttle, whether to
increase or decrease it to a certain value. This change in
acceleration creates a feedback to the sensor readings at the next
cycle. However, raw sensor values are often very noisy, so it is
usually not used directly to compute the throttle output. Instead, the
autopilot uses a Kalman Filter to fuse these raw sensor values to
obtain an estimate of position and heading that has less uncertainty.

%% \missingfigure{Basic sketch of a autopilot control system, with
%%   inputs, sensor fusion and feedback}

%% \missingfigure{A graph of position, altitude from raw sensor vs using
%%   a Kalman Filter}

\subsection{UAV Software Observations} \label{observations}

\noindent \textbf{Hardware Abstraction Layer (HAL) for common API} - In
terms of the software architecture, autopilots are constructed in
similar ways. Often there is a Hardware Abstraction Layer (HAL)
sitting between the program logic and the underlying hardware, which
presents a common API to the autopilot designer without having to
worry about lower level implementation details. This is due to the
fact that UAV usually have hardware components from different
manufacturers with different specifications and functionality.
~\\

\noindent \textbf{Repeated execution of tasks on a set interval} - All tasks have
a specific execution frequency depending on their importance. So tasks
such as reading altitude sensors execute at high rates such as 100Hz,
compared to updating the throttle which will occur only at 1Hz. Since
this is soft real-time, tasks are not necessarily forced to meet
deadline every time it executes. This is in opposition to
multithreading in normal software where execution of specific threads
are unpredictable (in most cases, first come first serve). Whereas
most if not all tasks in these systems have execution intervals
defined apriori.
~\\

\noindent \textbf{Stateful} - There is a frequent update of current state of the
vehicle, which is stored as either a global or a private
data structure. All information being passed is related to this
state. These variables are related to physical phenomenon, so these
variables must occur within a valid/safe envelop of operation.
~\\

\noindent \textbf{Different Scheduler Designs} - After looking at
Ardupilot and LibrePilot, there are two prevalent scheduler
designs. Each work as follows:

~\\
\noindent \textbf{Ardupilot}
\begin{enumerate}
\item Array of tasks, each with a particular interval
\item During each execution of “loop”
  \begin{enumerate}
    \item Wait for update sensor value from inertial
    sensors. Tries to read sensor, if value not available, delay
    100us. Repeat until value read or run out of time
    \item Executes the fast loop
    \begin{enumerate}
    \item Update heading
    \item Update outputs to motor
    \item Read inertial sensors
    \item Updates altitude controller
    \item Increment tick on scheduler
    \end{enumerate}
  \end{enumerate}
\item Scheduler then runs the tasks that it can within 2500 us.  Most
  tasks finish in 0 time, leaving a lot of time to finish the tick
\end{enumerate}

\noindent \textbf{Possible Contributions}

\begin{enumerate}
  \item Previous anomaly detection algorithms looked at program
    behavior indirectly by observing the variables in program
    predicates \cite{baah_2006} or built a model of program behavior
    using the sequence of function calls \cite{fei_argus_2006} or
    coming up with a list of potential invariants to monitor during
    program execution \cite{brun_finding_2004}.
    
    While each technique has its advantages, they all suffer from the
    fact that there is a huge
    space of variables to monitor. Whereas we explicitly
    make use of control-flow and data-flow analysis during compilation
    to narrow down the locations and variables to monitor.
  \item There is a dearth of good benchmarks from real autopilots in
    the literature, the best set of benchmarks is from JPappaBench,
    which was converted to do WCET analysis
\end{enumerate}

