\section{Discussion} \label{sec:discussion}

%% Description of why the results are meaningful. Were you able to
%% accomplish the goals you described in intro. What future work might
%% be done to improve results
The results show that each filter has its advantanges and
disadvantages. If we wanted to minimize average absolute error, then
the Particle Filter is the optimal solution with an average estimator
error of roughly 0.5 feet. However, the trade-off of low error is long
execution time per estimation step. Since the Kalman Filter can
execute over 400 times faster compared to the sampling frequency, this
means that it is more amenable to a real-time application as sensor
values will require less buffering. In fact, we can potentially
decrease the estimation error further if we run multiple Kalman
filters in parallel.

I can improve on this design by with additional IMU's so that the estimate
can improve or find ways to reduce the noise. One thing that has
\textit{not} been considered are the various errors that are
associated with each sensor. For instance, IMU's are known to have
problems of bias, drift and accumulated error. Similarly, even though
the barometer may have really high accuracy, sudden changes in
atmospheric pressure makes the sensor not as reliable of a source.
