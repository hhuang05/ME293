\section{Results} \label{sec:results}

\subsection{Tracking state variables}
Figure \ref{fig:kf} and \ref{fig:pf} shows one run of the simulation
scenario with the Kalman Filter and the Particle Filter tracking the
states after initial perturbation to altitude. Even though both
simulations had the same process noise model and the same gains for
the matrix $L$, the controller gave drastically different responses
for the simulations involving the Kalman Filter compared to the
Particle Filter.

For both estimators, we can see that combining the GPS sensor and the
Barometer sensor allows them to estimate altitude quite well. Both
Kalman and Particle filters have very little error in the altitude
estimate. However, both filters do not do well when it comes to
tracking the pitch angle change rate $q$. Both filters do slightly
better when it comes to the pitch angle $\theta$. 

\begin{figure}[h]
  \ig{scale=0.22}{kalman_20ft}
  \caption{The Kalman filter tracking the current state after a
    perturbation of 20 feet in altitude.}
  \label{fig:kf}
\end{figure}

\begin{figure}
  \ig{scale=0.2}{particle_20ft}
  \caption{The Particle filter tracking the current state of the
    aircraft after a perturbation of 20 feet in altitude. The
    estimate shown here is the mean of the particle cloud. }
  \label{fig:pf}
\end{figure}

\subsection{Comparing Mean Estimator Error}

Performance goal A stated in Section \ref{subsec:perfgoals} is to
identify an estimator with an average absolute error that is less than
10 feet in estimating altitude. Figure \ref{fig:kfpfcompare} shows a
comparison of the distributions of the average absolute errors of the
Kalman Filter and the Particle Filter. The figure has been normalized
with respect to the probability of observing an error. We can see that
the Particle Filter has very little average error, as the error
distribution is centered around 0.5 feet. In comparison, the Kalman
Filter has much more error as the the error distribution is centered
around 12 feet. This means that the Particle Filter \textit{meets} the
performance goal while the Kalman Filter \textit{does not meet} the
performance goal. 

\begin{figure}
  \ig{scale=0.22}{KF_PF_Meancompare}
  \caption{Histograms comparing the average absolute estimate error in
    altitude using estimates from 1000 runs of the Kalman filter and 1 run of
    the particle filter using the mean of 1000 particles.}
  \label{fig:kfpfcompare}
\end{figure}

\subsection{Comparing Time of Execution}

Performance goal B stated in Section \ref{subsec:perfgoals} is to have
the estimator complete its estimation task for a time step within 0.01
seconds. From Table \ref{table:time}, we can see that for each step,
the Kalman Filter executes over 400 times faster than the sampling
period of 0.01 seconds. The Particle Filter estimation is over
5 times slower than the sampling period. For performance goal B, the
Kalman Filter is the clear winner.

\begin{table}[h]
\centering % used for centering table
\begin{tabular}{c|c} % centered columns (4 columns)
  \hline \hline
  Filter & Avg Execution Time per Step \\
  \hline 
Kalman Filter & $2.19*10^{-5}$ sec \\
\hline 
Particle Filter & $5.64*10^{-2}$ sec \\
\hline %inserts single line
\end{tabular}
\caption{A comparison of the average execution time per step for the
  Kalman Filter and the Particle Filter with 1000 particles.} % title of Table
\label{table:time} % is used to refer this table in the text
\end{table}
