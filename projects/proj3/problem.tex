\section{Problem Description} \label{sec:problem}

%% Description of the problem you are trying to solve and the states you
%% are trying to control; description of the performance goal(s) you seek
%% to achieve through controller/estimator design. What is your
%% hypothesis?

\subsection{System Description}

Instead of applying state estimation techniques to a physical system,
I want to apply these techniques to a software-only
system. Specifically, looking at trying to apply state estimation
ideas to the insertion sort algorithm and data structure. Sorting is
the classical way introductory computer science students learn about
computational complexity. There are many sorting algorithms, we will
focus on the \textit{insertion} sort algorithm. We will illustrate this first
with an example and then we will give the exact algorithm. 

The algorithmm starts with an unordered list of numbers and starts at
the left end.



The system under study
The system that I am looking at is a software system, specifically
looking at trying to apply state estimation ideas to software. The
particular problem I am examining is insertion sort.

\begin{description}
\item[State] The state is the list of numbers to be sorted.
\item[Sensor Model] The hardest part of this project is coming up with a
  sensor model. Currently I am assuming perfect sensing into random
  elements of the sorted portion of the list.
\item[Prediction] State estimation methods such as the Kalman Filter
  generates a prior based on model dynamics, usually denoted $
  \hat{\mathbf{x}}_{k}^{+}$. For insertion sort, we will initially
  assume that the estimator is estimating the location to insert a new item
  in the sorted list. To produce this estimate, we will use the lever
  rule:
  \begin{align}
    f &= \frac{newval - lowval}{highval - lowval} \\
    i\_est &= (i\_high - i\_low)*f + i\_low
  \end{align}  
\item[Correction] To produce $\hat{\mathbf{x}}_{k}^{-}$, we are
  currently exploring two options. The first option is to use the
  sensor values and the ground truth and generate a residual, then we
  can use the residual to sort locally the place where the residual
  occurred. The second option is to do the lever rule again.
\end{description}

\subsection{Hypotheses} \label{subsec:hypo}

