\section{Problem Description} \label{sec:problem}

%% Description of the problem you are trying to solve and the states you
%% are trying to control; description of the performance goal(s) you seek
%% to achieve through controller/estimator design. What is your
%% hypothesis?

\subsection{System Description}

The problem of sorting numbers is a fundamental
way to study algorithms for introductory computer science
students. While there are many sorting algorithms, in this paper we
will focus on the \textit{insertion} sort algorithm. The insertion
sort algorithm sorts in place, which means that it needs no extra
storage to hold intermediate sorted lists. In order to sort in place,
the algorithm maintains an invariant, which is that the sorted
subarray always starts from the left and advances right. An example of
the insertion sort algorithm applied to a list of five numberes is
shown in Figure \ref{fig:insertsort}.

\begin{figure}[h]
  \ig{scale=0.8}{figs/insertsortexample}
  \caption{Example of insertion sort on a list of five numbers.}
  \label{fig:insertsort}
\end{figure}

In Figure \ref{fig:insertsort}, the red arrow points to the index of
the current item being examined. The black arrow points to the end of
the sorted subarray.  In step A, the current item is 5 and it is in
the right order. In step B, 1 is smaller than 3 and 5 so it moves to
the end of the sorted array. In step C, 9 is in the correct
position. In step D, 7 is less than 9 and so it must move down. In
step E, the algorithm completes as the black arrow advances to the end of the
array. 

The pseudocode for the insertion sort algorithm is shown in Algorithm 1. 

\begin{algorithm}
  \label{algo:insertsort}
  \begin{algorithmic}[1]
    \Procedure{insertsort}{$list$}
    \For{$i \gets 1$ to length($list$)}    
    \State $tmp \gets list \lbrack i \rbrack$
    \State $j \gets i - 1$
    \While{$j > 0$ and $tmp < list \lbrack j \rbrack$}
    \State $list \lbrack j+1 \rbrack \gets list \lbrack j \rbrack$
    \State $j \gets j - 1$
    \EndWhile
    \State $list \lbrack j+1 \rbrack \gets tmp$
    \EndFor
    \State \textbf{return} $list$
    \EndProcedure
  \end{algorithmic}
  \caption{The insertion sort algorithm assumes that the list is zero-indexed}  
\end{algorithm}



\begin{description}
\item[State] The state is the list of numbers to be sorted.
\item[Sensor Model] The hardest part of this project is coming up with a
  sensor model. Currently I am assuming perfect sensing into random
  elements of the sorted portion of the list.
\item[Prediction] State estimation methods such as the Kalman Filter
  generates a prior based on model dynamics, usually denoted $
  \hat{\mathbf{x}}_{k}^{+}$. For insertion sort, we will initially
  assume that the estimator is estimating the location to insert a new item
  in the sorted list. To produce this estimate, we will use the lever
  rule:
  \begin{align}
    f &= \frac{newval - lowval}{highval - lowval} \\
    i\_est &= (i\_high - i\_low)*f + i\_low
  \end{align}  
\item[Correction] To produce $\hat{\mathbf{x}}_{k}^{-}$, we are
  currently exploring two options. The first option is to use the
  sensor values and the ground truth and generate a residual, then we
  can use the residual to sort locally the place where the residual
  occurred. The second option is to do the lever rule again.
\end{description}

\subsection{Hypotheses} \label{subsec:hypo}

