\section{Analysis of Bugs ``in the Wild''} \label{fault_analysis}

\subsection{Prior work}

Evaluating bug detection algorithms has always been an issue, since
the program and the bug must be representative of those that exist in
the real world. The question of whether a program or a bug are
representative continue to persists as programs have become larger and
more complex. The initial basis was the Siemens bug suite created in
the early 90's, introduced by Hutchins et al \cite{hutchins1994}. This bug suite
contained seven C programs ranging from 141 to 512 lines long. Bugs were
artificially introduced, mostly by mutating single lines of code to
create 130 fault programs. The Siemens bug suite has been used
extensively in Bug detection \cite{fei_argus_2006} (other work).

The major concern in using the Siemens bug suite is that the programs
are generally quite short, all less than a thousand lines. This no
longer serves as the ``typical'' case in modern software. In addition,
the bugs were injected manually and are not representative of bugs
that arise naturally in software development projects. 

Other than using the Siemens bug suite, much work has focused on using
open-source software that has a collection of bugs reported. These
appear below:

\begin{enumerate}
\item Server and client applications
  \begin{enumerate}
    \item Concurrency bugs - MySQL, Apache, Mozilla and OpenOffice
      \cite{lu_learning_2008} \cite{fonseca_study_2010}
    \item Servers: Apache, Squid, Tomcat, OpenSSH, MySQL, SVN
      \cite{sahoo_empirical_2010} \cite{Li2006}
  \end{enumerate}
  \item Router Software \cite{yin_towards_2010}
  \item JPL and NASA unmanned space missions software
    \cite{grottke_empirical_2010}
  \item Database and Operating Systems : Tandem \cite{sullivan_software_1991}
    \cite{chillarege_orthogonal_1992}, Linux \cite{Tan2013}
\end{enumerate}

As can be seen from the above list, currently the literature lacks any
study of bugs in real-time control systems, although there is prior
work on server applications. There has been some work on generating
bug benchmarks for C/C++ applications \cite{cifuentes_begbunch_2009}
\cite{lu2005bugbench}. The work by Lu et al \cite{lu2005bugbench}
addressed mostly memory issues, with only two bugs that are semantic
bugs.

We hypothesize that because the requirements differ, the design for these
applications are fundamentally different, which lead to completely
different classes of bugs. If we can gain an understanding of the
classes of bugs and why they occur and how they occur in a particular
domain, we may have more genearlized knowledge to come up with a
solution.

Fundamental Question: Are the bugs encountered in UAV autopilots the
same classes of bugs that appear in open-source software as
investigated in prior work? If they are not the same distribution,
then the current tools that are used to tackle open-source software
may not be sufficient to detect bugs in avionics. 


\subsection{Summary of Bugs in UAV Autopilot Software}

\subsubsection{OpenPilot}

\textbf{**Note**}

Since the summer of 2015, there were disagreements within the project
team of OpenPilot, leading to the close of this project. As such, I
can no longer access the bug tracker and get the relevant details. The
project has a spin-off that is being actively developed called
Librepilot, but the bug tracker has no record of all the old bugs.

A listing of all the bugs that were manually classified appears in
Table \ref{table:opbugs}. 
 
\begin{table} [h]
  \begin{tabular}{r l}
    No. Bugs & Category \\
    \hline
    3 & Arithmetic Overflow \\
    2 & Bad Initial Value \\
    1 & Bad Task Priority \\
    22 & Build \\
    7 & Calculation Error \\
    1 & Category \\
    7 & Comms \\
    2 & Concurrency \\
    3 & Configuration \\
    1 & Connection \\
    4 & Downstream \\
    3 & External Lib \\
    3 & External System \\
    9 & Fail Safe \\
    32 & GUI \\
    1 & Integer Maxed \\
    1 & Integer Overflow \\ 
    13 & Logic \\
    11 & NA \\
    5 & Not Enough Information \\
    1 & Organization \\
    1 & Performance \\
    1 & Pointer Arithmetic \\
    5 & Programmer Mistake \\
    10 & Resource Bug \\
    4 & Saving \\
    1 & Saving and Sync \\
    9 & Wrong Default \\
    \hline
    \textbf{163} & Total \\
  \end{tabular}
  \caption{A summary of 163 manually classified bugs that appeared in
    Openpilot after filtering for priority Blocking and Major}
  \label{table:opbugs}
\end{table}

\subsubsection{Ardupilot}

We applied the Orthogonal Defect Classification (ODC) method 
\cite{chillarege_orthogonal_1992}, a well-regarded standard for
classifying bugs and their triggers to bugs in Ardupilot. We examined
bugs in Ardupilot pertaining specifically to Arducopter and narrow
down the field by applying filters to look at only bugs which were
legitimate bugs and fixed, not issues which are feature requests or
issues which won't be fixed. We have examined 35 bugs currently, and
have classified them manually according to the ODC. 

\begin{table} [h]
  \begin{tabular}{c|c c}
    DefectType & Count & Percentage \\
    \hline
    Function/Class/Object & 2 & 5.88\%\\
    Algorithm/Method & 8 & 23.53\% \\
    Assignment/Initialization & 13 & 38.24\%\\
    Checking & 6 & 17.65\%\\
    Interface/O-O Messages & 1 & 2.94\% \\
    Timing/Serialization & 4 & 11.76\%\\
    \hline
  \end{tabular}
\end{table}

If we look at the combination of Algorithm and Timing bugs, they make
up approximately 34\% of the bugs examined in the database. These are
the bugs that we are interested in. If we then drill down further and
check the breakdown in terms of their impact, we can see that they
impact largely in terms of reliability and capability, that is that
these bugs usually lead the system to become less reliable by taking
away or making the user suspicious of certain capability. 

From this analysis, we can then focus on algorithmic and timing bugs
as classified by the ODC meethod. Algorithmic bugs have the
characteristic that they do not immediately cause the program the
stop, unlike a memory errors like segmentation fault. Nor are they
directly concurrency related. They have many root causes, one of which
may be due to the nondeterministic nature of events generated by the
system.

\begin{table} [h]
  \begin{tabular}{r|c c}
    Functional Aspects & Algorithm/Method & Timing/Serialization \\
    \hline 
    Sensing & 2 & 2 \\
    Control & 1 & 2 \\
    Actuation & 0 & 0 \\
    Navigation & 2 & 0 \\
    Communication & 0 & 0 \\
    Overhead & 0 & 0 \\
    \hline
  \end{tabular}
\end{table}

\begin{table} [h]
  \begin{tabular}{c|c c}
    Impact & Algorithm/Method & Timing/Serialization \\
    \hline
    Reliability & 3 & 3 \\
    Capability & 3 & 0 \\
    Performance & 1 & 1 \\
    Requirements & 0 & 0 \\
    Installability & 0 & 0\\
    Usability & 1 & 0 \\
    \hline
  \end{tabular}
\end{table}

