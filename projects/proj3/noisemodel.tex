\section{Noise Model} \label{sec:noise}

\subsection{Process Noise}

I assume that $\theta, z$ has no process noise and only $u, w, q$ are affected
by wind. I assume that the wind is larger in the lateral direction
than the fore-aft direction and that pitch models are very good. I
assume that the effect of wind acts much like white noise, in that
it has a Gaussian distribution that is centered at 0 and it has the
following standard deviations: $\sigma_{\dot u} = 5 \frac{ft}{sec^2}$,
$\sigma_{\dot w} = 10 \frac{ft}{sec^2}$, $\sigma_{\dot q} = 0.1745
\frac{crad}{sec^2}$. This noise model is expressed in the following
process noise covariance matrix:
\begin{equation} \label{eq:processnoise}
Q =
\begin{bmatrix}
  5^2 & 0 & 0\\
  0 & 10^2 & 0\\
  0 & 0 & 0.1745^2\\
\end{bmatrix} 
\end{equation}

\subsection{Sensor Noise}

The sensors include the following:
\begin{description}
\item [GPS] - This is included to estimate horizontal position $x$ and
  altitude $z$. Assuming white gaussian noise (WGN) with $\sigma = 5$ meters
  \cite{zaliva2014}
\item [IMU] - Since this is a commercial aircraft, I assume this is a
  high end IMU and that it suffers from WGN with $\sigma = 0.7$
  degrees per hour. (KVH.com white paper)
\item [Barometer] - With the GPS present, this is a redundant sensor,
  which derives altitude from pressure. In reality, barimetric pressure
  is easily affected by the atmosphere which leads to bad
  readings. So even though this sensor is a more sensitive, it is
  prone to drift. However, we will not be modeling this sensor drift
  and will only assume that the sensor is corrupted with normal
  gaussian noise. I assume that accuracy is better than GPS and
  $\sigma = 0.5$ meters. \cite{zaliva2014}
\end{description}

Therefore, the sensor noise is as follows for GPS, IMU and Barometer respectively:
\begin{equation} \label{eq:sensornoise}
R =
\begin{bmatrix}
  (16.4 \text{ ft})^2 \\
  (1.22 \text{ crad})^2 \\
  (1.64 \text{ ft})^2
\end{bmatrix} 
\end{equation}
